\documentclass{minimal}

\usepackage{etoolbox}

\begin{document}
    \makeatletter

    \newcounter{num}
    \def\totalpoints{0}
    \def\totalpointsii{0}

    \newcommand{\CountRoundPoints}[2]
    {
        \ifdefstring{#1}{A}
        {
            \ifdefstring{#2}{X }
            {
                \edef\totalpoints{\the\numexpr\totalpoints+1+3\relax}
            }
            {}
            \ifdefstring{#2}{Y }
            {
                \edef\totalpoints{\the\numexpr\totalpoints+2+6\relax}
            }
            {}
            \ifdefstring{#2}{Z }
            {
                \edef\totalpoints{\the\numexpr\totalpoints+3+0\relax}
            }
            {}
        }
        {}
        \ifdefstring{#1}{B}
        {
            \ifdefstring{#2}{X }
            {
                \edef\totalpoints{\the\numexpr\totalpoints+1+0\relax}
            }
            {}
            \ifdefstring{#2}{Y }
            {
                \edef\totalpoints{\the\numexpr\totalpoints+2+3\relax}
            }
            {}
            \ifdefstring{#2}{Z }
            {
                \edef\totalpoints{\the\numexpr\totalpoints+3+6\relax}
            }
            {}
        }
        {}
        \ifdefstring{#1}{C}
        {
            \ifdefstring{#2}{X }
            {
                \edef\totalpoints{\the\numexpr\totalpoints+1+6\relax}
            }
            {}
            \ifdefstring{#2}{Y }
            {
                \edef\totalpoints{\the\numexpr\totalpoints+2+0\relax}
            }
            {}
            \ifdefstring{#2}{Z }
            {
                \edef\totalpoints{\the\numexpr\totalpoints+3+3\relax}
            }
            {}
        }
        {}
    }

    \newcommand{\CountRoundPointsPartII}[2]
    {
        \ifdefstring{#1}{A}
        {
            \ifdefstring{#2}{X }
            {
                \edef\totalpointsii{\the\numexpr\totalpointsii+3+0\relax}
            }
            {}
            \ifdefstring{#2}{Y }
            {
                \edef\totalpointsii{\the\numexpr\totalpointsii+1+3\relax}
            }
            {}
            \ifdefstring{#2}{Z }
            {
                \edef\totalpointsii{\the\numexpr\totalpointsii+2+6\relax}
            }
            {}
        }
        {}
        \ifdefstring{#1}{B}
        {
            \ifdefstring{#2}{X }
            {
                \edef\totalpointsii{\the\numexpr\totalpointsii+1+0\relax}
            }
            {}
            \ifdefstring{#2}{Y }
            {
                \edef\totalpointsii{\the\numexpr\totalpointsii+2+3\relax}
            }
            {}
            \ifdefstring{#2}{Z }
            {
                \edef\totalpointsii{\the\numexpr\totalpointsii+3+6\relax}
            }
            {}
        }
        {}
        \ifdefstring{#1}{C}
        {
            \ifdefstring{#2}{X }
            {
                \edef\totalpointsii{\the\numexpr\totalpointsii+2+0\relax}
            }
            {}
            \ifdefstring{#2}{Y }
            {
                \edef\totalpointsii{\the\numexpr\totalpointsii+3+3\relax}
            }
            {}
            \ifdefstring{#2}{Z }
            {
                \edef\totalpointsii{\the\numexpr\totalpointsii+1+6\relax}
            }
            {}
        }
        {}
    }

    \newread\inputfile% declare read handle

    \openin\inputfile=input% open input for reading

    \loop\unless\ifeof\inputfile% loop until we reach eof
        \read\inputfile to\input% read current line into \input
        \setcounter{num}{0}%
        \@for\letter:=\input\do%
        {%
            \stepcounter{num}%
            \expandafter\edef\csname player\roman{num}\endcsname{\letter}%
        }%
        \CountRoundPoints%
        {\playeri}%
        {\playerii}%
        \CountRoundPointsPartII%
        {\playeri}%
        {\playerii}%
    \repeat%
    \typeout{=====================================}
    \typeout{Result part 1: \totalpoints}
    \typeout{Result part 2: \totalpointsii}
    \typeout{=====================================}

    \closein\inputfile

    \makeatother
\end{document}